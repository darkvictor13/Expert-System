\documentclass[12pt, a4paper]{article}
\usepackage{meu}
\usepackage{booktabs}
\usepackage{siunitx}

\renewcommand{\thefootnote}{\roman{footnote}}
\setlength{\headheight}{15pt}

\begin{document}
\capa%
\tableofcontents%
\listoffigures\cleardoublepage%

\section{Introdução}\label{sec:intro}

\section{Entrevista}\label{sec:entrevista}
\begin{itemize}
    \item Quais condições que impedem de fazer um financiamento?

    Resposta: Restrições no CPF, sendo elas dívidas no Serasa, regularização na receita federal. Ser menor de idade

    \item Normalmente qual o fluxo de perguntas a serem feitas para escolher a financeira?

    Resposta: Se a pessoa tem alguma preferência por algum banco, cliente que possui relacionamento com o banco possui menores taxas. Renda, a pessoa precisa ter capacidade de pagamento, informar o cliente dos gastos, IPVA, transferência..., parcela não pode ser 1/3 da renda.

    \item Quais as condições para financiar no Santander?

    Resposta: ter CNH, carro com menos de 20 anos de uso.

    \item Quais as condições para financiar na BV

    Resposta: Ter RG, CPF, carro com menos de 25 anos de uso.

    \item Quais as condições para financiar no Sicredi?

    Resposta: carro com menos de 10 anos de uso, ter CNH, veículo não pode ser de leilão.

    \item Quais as condições para financiar no Itaú?

    Resposta: ter CNH, menos de 20 anos, não pode ter passagem por leilão.

    \item Quais as condições para financiar no Banco Pan?

    Resposta: RG, CPF, 20 anos de uso

    \item Quais as condições para financiar na omni financeira?

    Resposta: carro com menos de 30 anos de uso, pessoa com mais de 21
\end{itemize}

\bibliography{ref}

\end{document}
