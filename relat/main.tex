\documentclass[12pt, a4paper]{article}
\usepackage{meu}
\usepackage{booktabs}
\usepackage{siunitx}

\renewcommand{\thefootnote}{\roman{footnote}}
\setlength{\headheight}{15pt}

\begin{document}
\capa%
\tableofcontents%
\listoffigures\cleardoublepage%

\section{Introdução}\label{sec:intro}
Financiamento ``é o ato de uma organização,
usualmente uma empresa,
que ajuda a pagar um produto ou um serviço de uma pessoa,
ou de outra empresa,
através de doação de dinheiro ou empréstimo\cite{financiamento}''.

\subsection{Especialista}

\section{Desenvolvimento do sistema especialista}\label{sec:desenvolvimento}
\subsection{Pesquisa a priori}
Antes de iniciar as entrevistas foi realizada uma pesquisa a respeito do mercado de financiamento de veículos no Brasil,
pesquisa essa facilitada devido ao parentesco com a especialista.
Sendo descoberto que existem 44 instituições financeiras que fazem financiamento de veículos no Brasil\cite{taxas_juros_bcb},
essas instituições possuem taxas de juros baseadas na descrita no site do Banco Central do Brasil\cite{taxas_juros_bcb},
porém não é possível descobrir exatamente a quantidade de juros a ser paga pelo cliente antes de tentar aprovar o financiamento utilizando o \textit{softaware} da instituição financeira,
visto que fatores tais como renda, histórico de transações com a financeira, idade, tempo de trabalho, estado civil entre outros que podem ou não ser específicos de uma única instituição, são levados em consideração para aprovar o financiamento.
Dessa maneira constatou-se que não é possível fazer uma análise de qual instituição financeira possui a menor taxa de juros com o escopo do projeto.

Com a pesquisa também descobriu-se que dessas 44 instituições financeiras, 6 delas são parceiras da Almeida Carros, sendo elas:
\begin{itemize}
    \item Santander;
    \item BV;
    \item Sicredi;
    \item Itaú;
    \item Banco Pan;
    \item Omni Financeira.
\end{itemize}

Tendo o conhecimento de quais empresas são parceiras da Almeida Carros, facilitou a elaboração das perguntas a serem feitas na entrevista com a especialista.

\subsection{Primeira entrevista}
No dia 2 de Janeiro de 2022, foi realizada a primeira entrevista com a especialista, segue abaixo de forma escrita as perguntas e respostas faladas na entrevista.
\begin{enumerate}
    \item Quais condições que impedem de fazer um financiamento?

    \textbf{Resposta}: Restrições no CPF, sendo elas dívidas no Serasa, regularização na receita federal. Ser menor de idade

    \item Normalmente qual o fluxo de perguntas a serem feitas para escolher a financeira?

    \textbf{Resposta}: Se a pessoa tem alguma preferência por algum banco, cliente que possui relacionamento com o banco possui menores taxas. Renda, a pessoa precisa ter capacidade de pagamento, informar o cliente dos gastos, IPVA, transferência..., parcela não pode ser 1/3 da renda.

    \item Quais as condições para financiar no Santander?

    \textbf{Resposta}: ter CNH, carro com menos de 20 anos de uso.

    \item Quais as condições para financiar na BV

    \textbf{Resposta}: Ter RG, CPF, carro com menos de 25 anos de uso.

    \item Quais as condições para financiar no Sicredi?

    \textbf{Resposta}: carro com menos de 10 anos de uso, ter CNH, veículo não pode ser de leilão.

    \item Quais as condições para financiar no Itaú?

    \textbf{Resposta}: ter CNH, menos de 20 anos, não pode ter passagem por leilão.

    \item Quais as condições para financiar no Banco Pan?

    \textbf{Resposta}: RG, CPF, 20 anos de uso

    \item Quais as condições para financiar na omni financeira?

    \textbf{Resposta}: carro com menos de 30 anos de uso, pessoa com mais de 21
\end{enumerate}

\subsection{Montagem da árvore de decisão}

\section{Conclusão}\label{sec:conclusao}
TODO: Falar que foi testado com a especialista o sistema

\cleardoublepage
\bibliography{ref}

\end{document}
